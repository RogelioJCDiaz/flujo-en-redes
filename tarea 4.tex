\documentclass[a4paper, 11pt]{article}
\usepackage[spanish]{babel}
\selectlanguage{spanish}
\usepackage{graphicx}
\usepackage{wrapfig}
\usepackage[utf8]{inputenc}
\usepackage{amsmath}
\usepackage{dsfont}
\usepackage{multirow}
\usepackage{vmargin}
\usepackage{subfigure}
\usepackage[numbers, sort&compress]{natbib}
\usepackage{url}
\usepackage{cite}
\usepackage{wrapfig}
\usepackage{enumerate} 
\usepackage{sectsty} % centrar secciones de encabezados
\usepackage[usenames]{color}
\usepackage[dvipsnames]{xcolor}
\usepackage{fancyvrb}
\usepackage{caption}
\usepackage{amsfonts}
\usepackage{amssymb}
\usepackage{listings}
\usepackage{color}
\usepackage{algpseudocode}
\usepackage{multirow}
\usepackage[usenames]{color}
\usepackage{epstopdf}
\usepackage{float}
\usepackage{nameref}
\spanishdecimal{.}
\title{Tarea 4}
\date{\today}
\author{1985269}




\begin{document}
\maketitle

\section{Descripci\'on del experimento}

Se escogieron tres generadores de grafos de la librer\'ia NetworkX, con cada un de ellos se generaron cuatro grafos de distinto orden (logar\'itmico de base 3). De cada orden se obtuvieron diez grafos con densidades distintas. Luego se determin\'o el flujo m\'aximo de cada grafo con cinco combinaciones de fuente sumidero distintas cinco veces m\'as. Para calcular el flujo m\'aximo se utilizaron tres algoritmos, para cada uno de ellos se realizo el procedimiento anterior.

Se almacenan los tiempos de ejecuci\'on de cada uno de los ciclos con el objetivo de estudiar los factores que lo afectan.  Otras variables de inter/'es estudiadas son: algoritmo, generador del grafo, cantidad de nodos y densidad.

Los algoritmos utilizados fueron:
\begin{itemize}
    \item Maximum flow
    \item Edmonds Karp
    \item Boykov Kolmogorov}
\end{itemize}

Generadores de Grafos:
\begin{itemize}
    \item dense gnm random graph
    \item gnm random graph
    \item gnp random graph}
\end{itemize}



El objetivo planteado es determinar si las  variables de inter\'es influyen en el tiempo de ejecuci\'on, para lo cual se realiz\'o un an\'alisis de varianzas y uno de coorrelaci\'on.   


A continuaci\'on se comparte el c\'odigo de Python con el que se recopil\'o la informaci\'on:
\newpage

\definecolor{mygreen}{rgb}{0,0.6,0}
\definecolor{mygray}{rgb}{0.5,0.5,0.5}
\definecolor{mymauve}{rgb}{0.58,0,0.82}

\lstset{ 
  backgroundcolor=\color{white},   % choose the background color; you must add \usepackage{color} or \usepackage{xcolor}; should come as last argument
  basicstyle=\footnotesize,        % the size of the fonts that are used for the code
  breakatwhitespace=false,         % sets if automatic breaks should only happen at whitespace
  breaklines=true,                 % sets automatic line breaking
  captionpos=b,                    % sets the caption-position to bottom
  commentstyle=\color{mygreen},    % comment style
  deletekeywords={...},            % if you want to delete keywords from the given language
  escapeinside={\%*}{*)},          % if you want to add LaTeX within your code
  extendedchars=true,              % lets you use non-ASCII characters; for 8-bits encodings only, does not work with UTF-8
  firstnumber=1,                % start line enumeration with line 1000
  frame=single,	                   % adds a frame around the code
  keepspaces=true,                 % keeps spaces in text, useful for keeping indentation of code (possibly needs columns=flexible)
  keywordstyle=\color{blue},       % keyword style
  language=Octave,                 % the language of the code
  morekeywords={*,...},            % if you want to add more keywords to the set
  numbers=left,                    % where to put the line-numbers; possible values are (none, left, right)
  numbersep=5pt,                   % how far the line-numbers are from the code
  numberstyle=\tiny\color{mygray}, % the style that is used for the line-numbers
  rulecolor=\color{black},         % if not set, the frame-color may be changed on line-breaks within not-black text (e.g. comments (green here))
  showspaces=false,                % show spaces everywhere adding particular underscores; it overrides 'showstringspaces'
  showstringspaces=false,          % underline spaces within strings only
  showtabs=false,                  % show tabs within strings adding particular underscores
  stepnumber=1,                    % the step between two line-numbers. If it's 1, each line will be numbered
  stringstyle=\color{mymauve},     % string literal style
  tabsize=2,	                   % sets default tabsize to 2 spaces
  title=\lstname                   % show the filename of files included with \lstinputlisting; also try caption instead of title
}
\begin{lstlisting}[frame=single]
Graf=nx.Graph()
rog=0
diccionario_inst_Shortest_path={}
while rog<=4:
    Graf.clear()
    rango=random.randint(random.randint(10,len(matrizadd)),len(matrizadd))
    for i in range(rango):
        for j in range(rango):
            if matrizadd[i,j]!=0:
                matrizadd[j,i]=0
                #Graf.add_weighted_edges_from([(i,j,matrizadd[i,j])])
                Graf.add_edges_from([(i,j)])
    cont=0
    lista_tiempos=[] 
    lista_tiempos_completos={}    
    tiempo_de_paro=0
    tiempo_inicial=0
    tiempo_final =0 
    tiempo_ejecucion=0    
    for r in range(30):
        lista_tiempos_completos[r+1]=[]
        tiempo_de_paro=0    
        while tiempo_de_paro<1:            
            tiempo_inicial = time()             
            nx.shortest_path(Graf, source=None, target=None, weight=None, method='dijkstra')           
            tiempo_final = time()  
            tiempo_ejecucion = tiempo_final - tiempo_inicial        
            if tiempo_ejecucion>0.0:
                lista_tiempos_completos[r+1].append((tiempo_ejecucion*10000))  
            tiempo_de_paro+=tiempo_ejecucion  
    guardar_n_e[rog]=[]
    diccionario_inst_Shortest_path[rog]=[]
    for i in lista_tiempos_completos.keys():
         media=np.mean(lista_tiempos_completos[i])
         diccionario_inst_Shortest_path[rog].append(media)
    guardar_n_e['nodos'].append(len(Graf.nodes))
    guardar_n_e['edges'].append(len(Graf.edges))  
    guardar_n_e['media'].append(np.mean(diccionario_inst_Shortest_path[rog]))
    guardar_n_e['desv'].append(np.std(diccionario_inst_Shortest_path[rog]))
    rog+=1 
\end{lstlisting}

\section{ANOVA}
Se realiz\'o un an\'alisis de varianzas para cada uno de las variables con el objetivo de determinar si la medias con respecto al tiempo son diferentes. Como es posible observar en las salidas de la prueba el valor de P es muy pequeño, por lo que se puede afirmar que las medias de las variables con respecto al tiempo son diferentes. Entonces es posible concluir que las variables se relacionan con el tiempo de ejecuci\'on.
\newpage

\RecustomVerbatimCommand{\VerbatimInput}{VerbatimInput}%
{fontsize=\footnotesize,
 frame=lines,  % top and bottom rule only
 framesep=2em, % separation between frame and text
 rulecolor=\color{Gray},
 label=\fbox{\color{Black}ANOVA.txt},
 labelposition=topline,
 commandchars=\|\(\), % escape character and argument delimiters for
                      % commands within the verbatim
 commentchar=*        % comment character
}
\VerbatimInput{ANOVA.txt}

\section{Mínimos cuadrados ordinarios}
Para estudiar m\'as a fondo la relaci\'on entre las variables y el tiempo se realiz\'o  una prueba de mínimos cuadrados ordinarios  (OLS) por sus siglas en ingl\'es. La salidas se muestran a continuaci\'on y el de R-squred indica que con estas variables es posible crear un modelo que elimine el setenta y cinco por ciento de los errores para determinar el tiempo. Del an\'alisis de los P valores se puede reafirmar que la medias de las variables con respecto al tiempo son diferentes.  La densidad es la m\'as influyente seg\'un este an\'alisis  seguida por el generdor y el orden, que aunque es menor se puede afirmar su relaci\'on con m\'as confiabilidad que la de la variable algoritmo ya que esta tiene menor P valor.
Adem\'as el an\'alisis arroja que podr\'ia existir una fuerte multicolinealidad.


\RecustomVerbatimCommand{\VerbatimInput}{VerbatimInput}%
{fontsize=\footnotesize,
 frame=lines,  % top and bottom rule only
 framesep=2em, % separation between frame and text
 rulecolor=\color{Gray},
 label=\fbox{\color{Black}OLS.txt},
 labelposition=topline,
 commandchars=\|\(\), % escape character and argument delimiters for
                      % commands within the verbatim
 commentchar=*        % comment character
}
\VerbatimInput{OLS.txt}


\section{Correlaci\'on y Comportamiento }
En la figura \ref{1} se muestra la matriz de correlaci\'on. Es posible observar que las correlaciones  con el tiempo de las variables orden y densidad ratificando la informaci\'on de la prueba anterior.
Por \'utimo en la gr\'afica \ref{2} se muestran en colores distintos cada uno de los algoritmos estudiados y puede observarse como a medida que aumenta el orden el tiempo crece abruptamente. Cada uno de los algoritmos fue representado con distintas formas.


\begin{figure}[H]
\centering
\includegraphics [width=80mm] {cor.eps}
\caption{Correlaci\'on}
\label{1}
\end{figure}



\begin{figure}[H]
\centering
\includegraphics [width=80mm] {scater.eps}
\caption{Comportamiento con respecto al tiempo}
\label{2}
\end{figure}



\bibliographystyle{unsrt}
\bibliography{nuevo}
\nocite{*}
\end{document}
