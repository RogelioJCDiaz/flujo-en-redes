\documentclass{article}

\usepackage[spanish]{babel}
\usepackage{listings}
\usepackage{epsfig} 
\usepackage{hyperref}
\usepackage{color}
%\usepackage{multicol}
%\usepackage{url}
%\renewcommand{\UrlFont}{\small}
\setlength{\parskip}{1cm}
\setlength{\parindent}{0pt}
\title{Prueba}
\date{29 de enero 2019}
\author{Rogelio Jes\'us Corrales D\'iaz}
%\institute{PISIS UANL}
%\email{dicoro993@gmail.com}



\begin{document}
\maketitle


\section{Introducción}

Este reporte pretende realizar un análisis de los principales tipos de grafos así como sus aplicaciones, y características principales. Para esto comenzaremos definiendo que es un grafo:

Un grafo G se define como un par (V,E), donde V es un conjunto cuyos elementos son denominados vértices o nodos y E es un subconjunto de pares no ordenados de vértices y que reciben el nombre de aristas o arcos.

Existen distintos tipos de grafos cada uno con sus especificidades. Atendiendo a sus características pudieran clasificarse de forma mas general como grafos simples o multigrafos.

Un grafo simple es aquel en el que para cada nodo U existe solo una arista que lo conecta con otro nodo V. Mientras que en un multigrafo dos vértices pueden esta conectados por mas de una arista.

Otra de las clasificación entre los grafos que resulta muy importante destacar, es si hay presencia de ciclos. Un grafo es acíclico cuando no es posible encontrar un camino tal que si comenzamos el recorrido en el nodo U (origen) el camino escogido nos regrese al dicho origen.

Esta ultima característica es bien importante ya que de aparecer ciclos en una grafo, encontrar la solución del problema que representa se complejiza.

Grafos dirigidos son aquellos donde la arista A que une dos vértices U y V se encuentra orientada en un sentido fijo de manera tal que el flujo que pasa por el vértice solo puede ir en el sentido previamente definido. En caso de que dicha dirección no sea estricta y los nodos sean unidos por una sola arista el grafo se clasifica como no dirigido.

Por último un grafo reflexivo es aquel donde existe una arista que conecta un vértice con si mismo.

En el documento ser\'a mostrado el c\'odigo de Python para generar cada uno de los grafos explicados. Fue utilizada la librer\'ia Networkx. De esta librer\'ia fueron usadas funciones para generar los grafos:
Graf para los simples, MutiGraf para los multigrafos, Digraf para los grafos dirigidos y MultiDigraf para los multigrafos dirigidos. Adem\'as los nodos reflexivos fueron resaltado en color rojo. 
 
 

\section{Grafo simple no dirigido acíclico}

En la figura \ref{uni} de la página \pageref{uni} se muestra un grafo que representa un problema de un arbol de expancion. Es evidente que no existen ciclo ya que no hay forma de partir de un nodo y escoger un recorrido (Conjunto de aristas) que nos regrese al origen. Este grafo puede representar un arbol de expancion de una red eléctrica. Donde el nodo origen es el numero 1 y este va alimentando al dos y al tres que a su vez alimentan a otros.

\definecolor{mygreen}{rgb}{0,0.6,0}
\definecolor{mygray}{rgb}{0.5,0.5,0.5}
\definecolor{mymauve}{rgb}{0.58,0,0.82}

\lstset{ 
  backgroundcolor=\color{white},   % choose the background color; you must add \usepackage{color} or \usepackage{xcolor}; should come as last argument
  basicstyle=\footnotesize,        % the size of the fonts that are used for the code
  breakatwhitespace=false,         % sets if automatic breaks should only happen at whitespace
  breaklines=true,                 % sets automatic line breaking
  captionpos=b,                    % sets the caption-position to bottom
  commentstyle=\color{mygreen},    % comment style
  deletekeywords={...},            % if you want to delete keywords from the given language
  escapeinside={\%*}{*)},          % if you want to add LaTeX within your code
  extendedchars=true,              % lets you use non-ASCII characters; for 8-bits encodings only, does not work with UTF-8
  firstnumber=1,                % start line enumeration with line 1000
  frame=single,	                   % adds a frame around the code
  keepspaces=true,                 % keeps spaces in text, useful for keeping indentation of code (possibly needs columns=flexible)
  keywordstyle=\color{blue},       % keyword style
  language=Octave,                 % the language of the code
  morekeywords={*,...},            % if you want to add more keywords to the set
  numbers=left,                    % where to put the line-numbers; possible values are (none, left, right)
  numbersep=5pt,                   % how far the line-numbers are from the code
  numberstyle=\tiny\color{mygray}, % the style that is used for the line-numbers
  rulecolor=\color{black},         % if not set, the frame-color may be changed on line-breaks within not-black text (e.g. comments (green here))
  showspaces=false,                % show spaces everywhere adding particular underscores; it overrides 'showstringspaces'
  showstringspaces=false,          % underline spaces within strings only
  showtabs=false,                  % show tabs within strings adding particular underscores
  stepnumber=1,                    % the step between two line-numbers. If it's 1, each line will be numbered
  stringstyle=\color{mymauve},     % string literal style
  tabsize=2,	                   % sets default tabsize to 2 spaces
  title=\lstname                   % show the filename of files included with \lstinputlisting; also try caption instead of title
}
\begin{lstlisting}[frame=single]
import networkx as nx
import matplotlib.pyplot as plt
G=nx.Graph()
G.add_edges_from([(1,2),(1,3), (3,4),(3,5),(4,10),(5,8),(2,6), (6,11), (2,7), (7,12)]) 
nx.draw(G, with_labels=True, pos=nx.spring_layout(G), edge_color='black', node_color='gray',node_size=1000, edgecolors='black', font_weight='bold')
G.number_of_nodes() 
plt.savefig('grafoNAS.eps', format='eps', dpi=1000)
\end{lstlisting}

\begin{figure}[h]
\centering
\includegraphics [width=80mm] {Primero.eps}
\caption{Grafo no dirigido acíclico}
\label{uni}
\end{figure}


\section{Grafo simple no dirigido cíclico}
La figura \ref{2} de la página \pageref{2} representa un grafo que puede ajustarse a un problema ampliamente conocido y estudiado, el problema del agente viajero. Este problema consiste en recorrer cada uno de los nodos una vez, minimizando la distancia total. Este es resulta un problema de optimizaci\'on combinatoria y esta comprendido dentro del grupo NP hard. Esto muestra a lo que hac\'iamos referencia cuando se enunci\'o que la existencia de ciclos complejiza el problema. Ya que en problemas como estos aunque el numero de soluciones factibles son finitas con el crecimiento del numero de nodos se hace improbable encontrar un algoritmo que arroje una solución en tiempo polinomial. 

  
\begin{lstlisting}[frame=single]
G=nx.Graph()
G.add_edges_from([(1,2),(2,3), (3,4),(4,5),(5,6),(6,7),(7,8), (8,1)]) 
\end{lstlisting}
\begin{figure}[h]
\centering
\includegraphics [width=80mm] {segundo.eps}
\caption{Grafo no dirigido cíclico}
\label{2}
\end{figure}


\section{Grafo simple no dirigido reflexivo}
En este caso se representan los nodos reflexivos con el color rojo, este grafo podr\'ia representar un grupo de tareas que se complementan entre si y no importa el orden en que se realicen. Adem\'as ademas las tareas representadas por los nodos reflexivos cuentan con un reproceso en caso de que sea necesario garantizar mayor calidad en el producto final. Figura \ref{3} en la pagina \pageref{3}.

\begin{lstlisting}[frame=single]
import networkx as nx
import matplotlib.pyplot as plt
G=nx.Graph()
G.add_edges_from([(1,2), (1,1), (2,3), (3,4), (4,1), (1,3), (2,4)]) 
node1 = {1,2}
node2 = {3,4}
pos = {1:(200, 350), 2:(550,350), 3:(650, 220), 4:(400,100), 5:(150,220)}
nodes=nx.draw_networkx_nodes(G, pos, with_labels=True, nodelist=node1,node_size=400, node_color='r', node_shape='o')
nodes=nx.draw_networkx_nodes(G, pos, with_labels=True, nodelist=node2,node_size=400, node_color='grey', node_shape='o')
nodes=nx.draw_networkx_edges(G, pos)
nx.draw_networkx_labels(G, pos)
plt.savefig('tercero con numeros.eps', format='eps', dpi=1000)
\end{lstlisting}
\begin{figure}[h]
\centering
\includegraphics [width=80mm] {tercero.eps}
\caption{Grafo no dirigido reflexivo}
\label{3}
\end{figure}

\section{Grafo simple dirigido acíclico}
Con este tipo de grafos pueden ser representados redes de distribuci\'on el\'ectrica comenzando por un nodo principal que alimenta a un grupo de nodos secundarios que a su vez alimentan a otros. Figura \ref{4} p\'agina \pageref{4}  
\begin{lstlisting}[frame=single]
G=nx.DiGraph()
G.add_edges_from([(1,2),(2,4), (2,3), (2,5), (5,6), (5,7)]) 
\end{lstlisting}
\begin{figure}[h]
\centering
\includegraphics [width=80mm] {cuarto.eps}
\caption{Grafo dirigido acíclico}
\label{4}
\end{figure}

\section{Grafo simple dirigido cíclico}
El grafo mostrado en la figura \ref{5} p\'agina \pageref{5} pudiera corresponder a una red de rutas de una ciudad donde tomando la primera ruta ser\'ia posible llegar a las rutas 2 y 3 y de esta forma como expresa la figura. Como podemos observa los v\'ertices se encuentran unidos por aristas que tienen una direcci\'on esto hace se reduzcan las variantes de mover flujo por el grafo.

\begin{lstlisting}[frame=single]
G=nx.DiGraph()
G.add_edges_from([(1,2),(2,3), (3,4), (1,3), (4,1)]) 
\end{lstlisting}


\begin{figure}[h]
\centering
\includegraphics [width=80mm] {quinto.eps}
\caption{Grafo no dirigido acíclico}
\label{5}
\end{figure}

\section{Grafo simple dirigido reflexivo}
Imaginemos que tenemos una linea de producci\'on con N procesos que quedan representados cada uno por un nodo, se conoce que los procesos 1 y 2 son obligatorios. Pero para obtener el producto deseado es necesario continuar  entonces debe ser tomada la decisi\'on de continuar con el proceso 3 o el 6 y asi de esta manera hasta recorrer todos los nodos. Figura \ref{6} p\'agina \pageref{6}.

\begin{lstlisting}[frame=single]
G=nx.DiGraph()
G.add_edges_from([(1,2),(2,3), (3,4)])
node1 = {1,2}
node2 = {3,4}
pos = {1:(200, 350), 2:(550,350), 3:(650, 220), 4:(400,100), 5:(150,220)}
nodes=nx.draw_networkx_nodes(G, pos, with_labels=True, nodelist=node1,node_size=400, node_color='r', node_shape='o')
nodes=nx.draw_networkx_nodes(G, pos, with_labels=True, nodelist=node2,node_size=400, node_color='grey', node_shape='o')
nodes=nx.draw_networkx_edges(G, pos)
nx.draw_networkx_labels(G, pos)
plt.savefig('tercero con numeros.eps', format='eps', dpi=1000) 
\end{lstlisting}
\begin{figure}[h]
\centering
\includegraphics [width=80mm] {sexto.eps}
\caption{Grafo simple dirigido reflexivo}
\label{6}
\end{figure}

\section{Multigrafo no dirigido acíclico}

Pudiéramos interpretar este grafo figura \ref{7} p\'agina \pageref{7} como la conexi\'on entre calles, donde existen dos maneras distintas de unirlas a excepci\'on del nodo 4. 

\begin{lstlisting}[frame=single]
G=nx.MultiGraph()
G.add_edges_from([(1,2), (2,1), (2,3,), (3,2,), (3,4,)]) 
\end{lstlisting}
\begin{figure}[h]
\centering
\includegraphics [width=80mm] {septimo.eps}
\caption{Multigrafo no dirigido acíclico}
\label{7}
\end{figure}


\section{Multigrafo no dirigido cíclico}
Imaginemos que tenemos representado cuatro procesos que generan informaci\'on y que la informaci\'on generada por cada uno debe ser compartida. Con este fin se conoce que de un proceso a otro (nodos) existen dos variantes de enviar la informaci\'on. Ahora bien cada variante tiene un tiempo de transmici\'on distinto. El objetivo definido por los decisores ser\'aminimizar este tiempo. 
Este problema queda representado por el grafo del la figura \ref{8} en la pagina \pageref{9}. Donde los nodos representan cada uno de los procesos y las aristas las variantes.

\begin{lstlisting}[frame=single]
G=nx.MultiGraph()
G.add_edges_from([(1,2), (2,1), (2,3,), (3,2,), (3,4,), (4,1), (1,4),(4,2),(2,4),(3,1),(1,3)]) 
\end{lstlisting}
\begin{figure}
\centering
\includegraphics [width=80mm] {octavo.eps}
\caption{Multigrafo no dirigido cíclico}
\label{8}
\end{figure}


\section{Multigrafo no dirigido reflexivo}
Ahora imaginemos que nos encontramos en una situaci\'on parecida que la secci\'on anterior. Pero resulta que los procesos 1 y 2 necesitan la infromaci\'on que ellos mismos generan para realizar controles. Y anteriormente la mandaban a los otros procesos para que fuera procesada. Si se valorara la opci\'on de que esta fuera estudiada en el mismo proceso con un tiempo de procesamiento determinado. Estaríamos en presencia del problema representado por el grafo de la figura \ref{9} en la pagina \pageref{9}. 

\begin{lstlisting}[frame=single]
G=nx.MultiGraph()
G.add_edges_from([(1,2),(1,1),(2,1),(2,2),(2,3,), (3,2,), (3,4,), (4,1), (1,4)]) 
\end{lstlisting}
\begin{figure}
\centering
\includegraphics [width=80mm] {noveno.eps}
\caption{Multigrafo no dirigido acíclico}
\label{9}
\end{figure}




\section{Multigrafo dirigido acíclico}
Supongamos que una empresa cuenta con 3 almac\'en desde el cual tiene que transportar las mercanc\'ias a dos clientes, pero antes esta debe pasar por controles de calidad para lo cual es transportada hacia la sede principal. Si sabemos que la capacidad de cada viaje es limitada y dependiente del n\'umero de veh\'iculos. Nos encontramos con un problema de transporte que esta representado por el grafo de la figura \ref{10} en la p\'agina \pageref{10}. Donde los nodos A=[4,5,6] S=[2] C=[1,3]  representan los almacenes, la empresa y los clientes respectivamente.  
\begin{lstlisting}[frame=single]
G=nx.MultiDiGraph()
G.add_edges_from([(1,2),(2,1),(2,3),(3,2),(4,2),(5,2), (6,2)])  
\end{lstlisting}

\begin{figure}
\centering
\includegraphics [width=80mm] {decimo.eps}
\caption{Multigrafo dirigido acíclico}
\label{10}
\end{figure}


\section{Multigrafo dirigido cíclico}
Retomemos ahora el problema del TSP, en este problema los nodos estan unidos por aristas que no necesariamente deben de estar dirigidas, ya que pueden ser usadas en ambas direcciones. Pero si a las posible le agregamos varias rutas que unen los destinos y a su vez estas tienen un solo destino, este problema es representado por un multigrafo dirigido c\'iclico figura \ref{11} pagina \pageref{11}. 

\begin{lstlisting}[frame=single]
G=nx.MultiDiGraph()
G.add_edges_from([(1,2),(2,1),(2,1),(2,3),(3,2),(3,2),(3,4),(4,3),(4,3),(4,1),(1,4),(4,2),(2,4),(3,1),(1,3)])) 
\end{lstlisting}
\begin{figure}
\centering
\includegraphics [width=80mm] {onceavo.eps}
\caption{Multigrafo no dirigido cíclico}
\label{11}
\end{figure}


\section{Multigrafo dirigido reflexivo}
Supongamos que una linea de producci\'on fabrica 4 componentes distintos pero solo puede ser producido un componente a la vez. Por esto cuando se desee fabricar otro hay que parar la producci\'on y cambiar la configuraci\'on lo cual conlleva un tiempo de ejecuci\'on. Ademas linea deber\'a parar cada ciertos periodos de tiempo por mantenimientos preventivos, y cada vez que pare se deber\'a re-configurar ya que fue reiniciada. Este problema se puede plantear como un TSP pero adem\'as como existe la posibilidad de que luego del mantenimiento se vuelva a usar la misma configuraci\'on se vuelve un problema representado por un Multigrafo dirigido reflexivo como el de la figura \ref{12} de la p\'agina \pageref{12}.

\begin{lstlisting}[frame=single]
G=nx.MultiDiGraph()
G.add_edges_from([(1,2), (2,1), (2,3), (3,2), (3,4), (4,3), (4,1), (1,4)])
\end{lstlisting}
\begin{figure}
\centering
\includegraphics [width=80mm] {doceavo.eps}
\caption{Multigrafo dirigido reflexivo}
\label{12}
\end{figure}

\bibliographystyle{ieeetr}
\bibliography{Bibliog}

\end{document}
